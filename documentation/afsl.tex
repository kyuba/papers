\documentclass[a4paper,twoside,titlepage]{article}
\usepackage[top=3cm,bottom=3cm,left=3cm,right=3cm]{geometry}
\usepackage{url}

\renewcommand\familydefault{\sfdefault}

\title{Advanced Filesystem Layout}
\author{Magnus Achim Deininger}


\begin{document}
\maketitle
\tableofcontents

\newpage

\section*{Preface}
\addcontentsline{toc}{section}{Preface}
Current filesystem layouts employed by contemporary operating systems suffer
from a number of shortcomings. Among the more common of these are general
inconsistencies and issues with contemporary multi-architecture systems. Minor,
similar issues occur when trying to develop for architectures other than that of
the host system. In general, it is becoming increasingly common that binaries
(executables and libraries) of distinct systems occur together on a single
system. Examples for these would include contemporary desktop systems with
``64-bit'' capabilities mixing binaries compiled for x86-64 and x86-32 on the
same filesystem with poor separation, or average Linux installations containing
binaries intended for Microsoft Windows for use with WINE, or FreeBSD
installations including binaries intended for Linux.

This has lead to what may be perceived as inconsistencies in order to fix some
of these issues, such as the lib/lib32/lib64 dilemma that would require sane
software build scripts to figure out in which of these directories to put
created libraries. This alone is already inconsistent with requirements that
expressly forbid this -32/-64 extension on bin/ directories\footnote{See the FHS
document for details.}, and the idea itself seems rather hackish.

Additionally, the original reasoning behind a split of the root filesystem and
the /usr hierarchy (and also somewhat the /opt and /usr/local hierarchies) seems
to have been to be able to separate machine-local binaries and configuration
from site-local binaries and shared data and read-only from read-write data.
Originally a physical separation between the hierarchies was necessary, but due
to the ability of contemporary operating systems to unify multiple filesystem
hierarchies, sometimes called "union mounts", this division seems quite obsolete
\footnote{Operating Systems such as Plan 9 seem to agree with this.}.

This paper will try to explore an alternative filesystem hierarchy scheme that
should provide consistency for years to come.
\newpage

\section{The Advanced Filesystem Layout}
The general idea behind this filesystem layout is fairly simple: "traditional"
directories are at the end of an arbitrarily long list of host specifications.
This leads to a general layout that looks something like this:
$os/architecture/subarchitecture/(bin|lib)$. The idea is to get more and more
specific along the path, with the operating system being considered less
specific than the hardware architecture. This latter choice is fairly random,
but it seems to be easier or more common to run binaries from a different
architecture on the same operating system than it is to run binaries from a
different operating system even when on the same architecture (although there
are legimitate examples of both).

\subsection{POSIX-compatibility considerations}

\newpage

\begin{thebibliography}{alpha}
\addcontentsline{toc}{section}{References}

\end{thebibliography}

\end{document}
