\documentclass[a4paper,twoside,titlepage]{article}
\usepackage[top=3cm,bottom=3cm,left=3cm,right=3cm]{geometry}
\renewcommand\familydefault{\sfdefault}
\title{Advanced Filesystem Layout}
\author{Magnus Achim Deininger}
\begin{document}
\maketitle
\section{Introduction}
Current filesystem layouts employed by contemporary operating systems suffer
from a number of shortcomings. Among the more common of these are general
inconsistencies and issues with contemporary multi-architecture systems. Minor,
similar issues occur when trying to develop for architectures other than that of
the host system. In general, it is becoming increasingly common that binaries
(executables and libraries) of distinct systems occur together on a single
system. Examples for these would include contemporary desktop systems with
``64-bit'' capabilities mixing binaries compiled for x86-64 and x86-32 on the
same filesystem with poor separation, or average Linux installations containing
binaries intended for Microsoft Windows for use with WINE, or FreeBSD
installations including binaries intended for Linux.
\end{document}
